\documentclass[11pt,a4paper]{scrartcl}
\usepackage[T1]{fontenc}
\usepackage[utf8]{inputenc}
\usepackage[ngerman]{babel}
\usepackage{microtype}
\usepackage{lmodern}
\usepackage{amsmath}
\usepackage{amsfonts}
\usepackage{amssymb}
\usepackage{enumerate}

\begin{document}

\author{Gruppe 14\\Max-Emanuel Hoffmann\\Ralf Vogler\\Sebastian Wiesner}
\title{Verteilte und Web-Informationssysteme}
\subtitle{Blatt 06}

\maketitle

\section{Aufgabe 1}
\subsection{Information Manifold}
\subsubsection{Globale Prädikate}
\begin{itemize}
\item car(C): C ist Auto
\item brand(C, B): B ist Marke von C
\item model(C, B, M): M ist Modell von Auto C der Marke B
\item price(C, P): P ist der Preis von C
\item distance(C, D): D sind die gefahrenen Kilometer von C
\item age(C, A): A ist das Jahr der Erstzulassung von C
\item fuel(C, F): F ist die Kraftstoffart von C
\item location(C, L): L ist der Verkaufsstandort von C
\item radius(C, L, R): R ist der Umkreis um den Verkaufsstandort L von C
\item warranty(C, W): W bestimmt ob C eine Garantie hat (true oder false)
\item new(C, N): N bestimmt ob C ein Neuwagen ist (true oder false)
\end{itemize}

\subsubsection{Sichten}
\paragraph*{autoscout24.de}
\begin{align*}
autoscout(B,M,P,D,A,F,L,R,W) &:- car(C) \& brand(C,B) \& model(C,B,M)\\
&\& price(C,P) \& distance(C,D) \& age(C,A)\\
&\& fuel(C,F) \& location(C,L) \& radius(C,L,R)\\
&\& warranty(C,W).
\end{align*}

\paragraph*{mobile.de}
Wenn nach Neuwagen gesucht wird, werden die Erstzulassung und Kilometerstand in der Suchmaske ausgeblendet. Deshalb sind zwei Regeln notwenig (Veroderung). 
\begin{align*}
mobile(B,M,P,D,A,F,L,R,N) &:- car(C) \& brand(C,B) \& model(C,B,M)\\
&\& price(C,P) \& distance(C,D) \& age(C,A)\\
&\& fuel(C,F) \& location(C,L) \& radius(C,L,R)\\
&\& new(C,N) \& N==false.
\end{align*}

\begin{align*}
mobile(B,M,P,D,A,F,L,R,N) &:- car(C) \& brand(C,B) \& model(C,B,M)\\
&\& price(C,P)\\
&\& fuel(C,F) \& location(C,L) \& radius(C,L,R)\\
&\& new(C,N) \& N==true.
\end{align*}


\subsection{Tsimmis}
\subsubsection{MSL Regeln}
\paragraph*{autoscout24.de}
\begin{verbatim}
<f(C) carSearch {
    <car C> <brand B> <model M>
    <price P> <distance D> <age A>
    <fuel F> <location L> <radius R>
    <warranty W>
}> @med :-
<carAutoscout {
    <car C> <brand B> <model M> 
    <price P> <distance D> <age A> 
    <fuel F> <location L> <radius R>
    <warranty W>
}>@autoscout24.de
\end{verbatim}

\paragraph*{mobile.de}
\begin{verbatim}
<f(C) carSearch {
    <car C> <brand B> <model M>
    <price P> <distance D> <age A>
    <fuel F> <location L> <radius R>
    <new N>
}> @med :-
<carMobile {
    <car C> <brand B> <model M> 
    <price P> <distance D> <age A> 
    <fuel F> <location L> <radius R>
    <new N>
}>@mobile.de
\end{verbatim}

\section{Aufgabe 2}

Mithilfe des \emph{HORIZ\_PART} Algorithmus sowie durch logische
Schlussfolgerung ergibt sich unter der Annahme gleichverteilter Zugriffe auf die
verschiedenen Relationen folgende Partitionierung:

\begin{align*}
  F_1 &= A_1 \wedge A_2 & | AbtNr \in \left[100 \ldots 150\right]\\
  F_2 &= A_2 \wedge A_5 & | AbtNr \in \left[151 \ldots 250\right]\\
  F_3 &= A_2 \wedge A_5 & | AbtNr \in \left[251 \ldots 299\right]\\
  F_4 &= A_2 \wedge A_6 & | AbtNr \in \left[300 \ldots 400\right]\\
  F_5 &= A_3 \wedge A_6 & | AbtNr \in \left[401 \ldots 499\right]
\end{align*}

Diese Partitionierung ist die möglichst granulare Partitionierung.  Es zeigt
sich, dass alle Prädikate relevant sind.

\end{document}
