\documentclass[11pt,a4paper]{scrartcl}
\usepackage[T1]{fontenc}
\usepackage[utf8]{inputenc}
\usepackage[ngerman]{babel}
\usepackage{microtype}
\usepackage{lmodern}
\usepackage{amsmath}
\usepackage{amsfonts}
\usepackage{amssymb}
\usepackage{enumerate}

\begin{document}

\author{Gruppe 14\\Max-Emanuel Hoffmann\\Ralf Vogler\\Sebastian Wiesner}
\title{Verteilte und Web-Informationssysteme}
\subtitle{Blatt 06}

\maketitle

\section*{Aufgabe 1}

\section*{Aufgabe 2}

Mithilfe des \emph{HORIZ\_PART} Algorithmus sowie durch logische
Schlussfolgerung ergibt sich unter der Annahme gleichverteilte Zugriffe auf die
verschiedenen Relationen folgende Partitionierung:

\begin{align*}
  F_1 &= A_1 \wedge A_2 & | AbtNr \in \left[100 \ldots 150\right]\\
  F_2 &= A_2 \wedge A_5 & | AbtNr \in \left[151 \ldots 250\right]\\
  F_3 &= A_2 \wedge A_5 & | AbtNr \in \left[251 \ldots 299\right]\\
  F_4 &= A_2 \wedge A_6 & | AbtNr \in \left[300 \ldots 400\right]\\
  F_5 &= A_3 \wedge A_6 & | AbtNr \in \left[401 \ldots 499\right]
\end{align*}

Diese Partitionierung ist die möglichst granulare Partitionierung.  Es zeigt
sich, dass alle Prädikate relevant sind.

\end{document}
