\documentclass[11pt,a4paper]{scrartcl}
\usepackage[T1]{fontenc}
\usepackage[utf8]{inputenc}
\usepackage[ngerman]{babel}
\usepackage{microtype}
\usepackage{lmodern}
\usepackage{amsmath}
\usepackage{amsfonts}
\usepackage{amssymb}
\usepackage{enumerate}

\begin{document}

\author{Gruppe 14\\Max-Emanuel Hoffmann\\Ralf Vogler\\Sebastian Wiesner}
\title{Verteilte und Web-Informationssysteme}
\subtitle{Blatt 5}

\maketitle

\section*{Aufgabe 1}

\begin{align*}
  ANGEST_1 &= \Pi_{PersNr,AngName,Anschrift}(ANGEST) \\
  ANGEST_2 &= \Pi_{PersNr,AbtNr}(ANGEST) \\
ANGEST_3 &= \Pi_{PersNr,AngName,Gehalt}(ANGEST)
\end{align*}

\section*{Aufgabe 2}

\begin{enumerate}[a)]
\item Die kanonische Übersetzung der Anfragen ist:
  \begin{align*}
    Anfrage_1 &= \Sigma_{A.a = B.b}(A \times C) \\
    Anfrage_2 &= \Sigma_{A.a = C.c}\left(\left(\Sigma_{A.b=500}(A)\right)
      \times C\right)
  \end{align*}
  Daraus ergeben sich folgende Operationen, die bei der Bestimmung der
  Allokation berücksichtigt werden müssen:
  \begin{enumerate}[I)]
  \item lese $p$
  \item lese $p$ mit Filter $A.b = 500$
  \end{enumerate}
\item Aus der Angabe und den Ergebnissen der vorherigen Teilaufgabe ergeben
  sich folgende Matrizen:
  \begin{align*}
    R_{tp} &=
    \begin{pmatrix}
      10000 & 10000 & 10000 \\
      9000 & 0 & 0
    \end{pmatrix} \\
    H_{it} &=
    \begin{pmatrix}
      210 & 10 \\
      190 & 150 \\
      430 & 30 \\
    \end{pmatrix}
  \end{align*}
\item Gemäß des Modells aus der Vorlesung muss folgende Summe minimiert werden,
  um die Allokationsmatrix $V_{pi}$ zu bestimmen:
  \begin{align*}
    \underbrace{\left(\sum_{p,j} G_p V_{pi} S_i\right)}_{\text{Speicherkosten }
   \sum_S} +
    \underbrace{\left(\sum_{i,t,p,j} H_{it} O_{tp} V_{pj} U_{ij} \right) +
    \left(\sum_{i,t,p,j}H_{it} R_{tp} V_{pj}
      U_{ji}\right)}_{\text{Übertragungskosten } \sum_U}
  \end{align*}
  Da $S_i$ gemäß Angabe ein Nullvektor ist, gilt $\sum_S = 0$.  Für den ersten
  Summanden aus $\sum_U$ kann ebenfalls $0$ angenommen werden, da bei
  Leseoperationen gemeinhin gilt: $R_{tp} \gg O_{tp}$.  Somit verbleibt zur Optimierung:
  \begin{align}
    \label{eq:min}
    \sum_{i,t,p,j}H_{it} R_{tp} V_{pj} U_{ji}
  \end{align}
  Da die Allokation gemäß Angabe redundanzfrei sein muss und jede Partition
  somit auf nur einem einzigen Knoten alloziert werden darf, gilt $\forall_{p
    \in [1,P]}:\left(\sum_i V_{pi}\right) = 1$.  Zudem darf eine Partition auf
  einem Knoten nur alloziert werden, wenn dieser Knoten genügend Speicherplatz
  bietet.  Es gilt also $\forall_{i \in [1,K]}\left(\sum_{p} G_p V_{pi}\right)
  \le M_i$.  Mit $M_i = \left(0, \infty, \infty\right)$ ergibt sich $V_{pi} =
  0$ für $i = 0$.  Unter Berücksichtigung dieser Randbedingungen lässt sich
  Gleichung~\ref{eq:min} für alle möglichen $V_{pi}$ berechnen (siehe
  \texttt{optimization.py}).  Es ergibt sich, dass die Summe für folgendes
  $V_{pi}$ minimal ist:
  \begin{align*}
    V_{pi} =
    \begin{pmatrix}
      0 & 0 & 1 \\
      0 & 0 & 1 \\
      0 & 0 & 1 \\
    \end{pmatrix}
  \end{align*}
  Die drei Partitionen werden also allesamt auf Station 3 alloziert.
\item Das Modell berücksichtigt Join-Operationen nicht, daher spielt auch deren
  Kardinalität keine Rolle.  Es ist egal, ob eine Relation vor dem Verbund
  übertragen wird oder das Ergebnis des Verbunds. Im ersten Fall gilt für die
  Anzahl der übertragenen Tupel $n = |A||B|$, da die Relation $A$ für jedes Tupel
  in $B$ übertragen werden muss.  Im zweiten Fall gilt $n = |A \times B| = |A||B|$.
\end{enumerate}


\section*{Aufgabe 3}
\begin{enumerate}[a)]
\item siehe Code
\item Threshold-Ranking ist ein Pipelinebreaker weil eine Relation eventuell komplett gelesen werden muss um den richtigen Partner zu finden. Gleiches gilt für das No-Random-Access-Ranking. Dabei werden die Relationen zwar sequentiell gelesen, allerdings kann es notwendig sein alle Tupel zwischenzuspeichern bevor die Ergebnisse weitergeleitet werden können.
\end{enumerate}


\end{document}
