\documentclass[11pt,a4paper]{scrartcl}
\usepackage[T1]{fontenc}
\usepackage[utf8]{inputenc}
\usepackage[ngerman]{babel}
\usepackage{microtype}
\usepackage{lmodern}
\usepackage{amsmath}
\usepackage{amsfonts}
\usepackage{amssymb}
\usepackage{enumerate}

\begin{document}

\author{Gruppe 14\\Max-Emanuel Hoffmann\\Ralf Vogler\\Sebastian Wiesner}
\title{Verteilte und Web-Informationssysteme}
\subtitle{Blatt 4}

\maketitle

\section*{Aufgabe 1}

\begin{align*}
  ABT_1 &= \sigma_{(100 \leq AbtNr \leq 220) \vee (AbtNr = 250)}(ABT) \\
  ABT_2 &= \sigma_{(221 \leq AbtNr \leq 370) \wedge (AbtNr \neq 250)}(ABT) \\
  ABT_3 &= \sigma_{(371 \leq AbtNr \leq 430)}(ABT) \\
\end{align*}

\section*{Aufgabe 2}

\begin{align*}
  ANGEST_1 &= ANGEST \ltimes_{PersNr=MgrPersNr} ABT \\
  ANGEST_2 &= ANGEST \ltimes (\Pi_{PersNr}(ANGEST) - \Pi_{PersNr=MgrPersNr}(
  ABT)) \\
\end{align*}

\section*{Aufgabe 3}

\begin{enumerate}
\item Die Partitionierung ist weder vollständig noch disjunkt.  Seien $R$ und
  $S$ wie folgt definiert:
  \begin{align*}
    R &=
    \begin{array}{c|c}
      \mathbf{A} & \mathbf{B} \\\hline
      a_1 & b_1 \\
      a_2 & b_1
    \end{array} & S &=
    \begin{array}{c|c}
      \mathbf{B} & \mathbf{C} \\\hline
      b_1 & c_1 \\
      b_2 & c_2
    \end{array}
  \end{align*}
  Sei $R$ disjunkt und vollständig partitioniert in $R_1$ und $R_2$ mit:
  \begin{align*}
    R_1 &=
    \begin{array}{c|c}
      \mathbf{A} & \mathbf{B} \\\hline
      a_1 & b_1
    \end{array}
    & R_2 &=
    \begin{array}{c|c}
      \mathbf{A} & \mathbf{B} \\\hline
      a_2 & b_1
    \end{array}
  \end{align*}
  Bei der abgeleiteten Partitionierung von $S$ ergibt sich dann:
  \begin{align*}
    S_1 &= S \ltimes R_1 =
    \begin{array}{c|c}
      \mathbf{B} & \mathbf{C} \\\hline
      b_1 & c_1
    \end{array} & S_2 &= S \ltimes R_2 =
    \begin{array}{c|c}
      \mathbf{B} & \mathbf{C} \\\hline
      b_1 & c_1
    \end{array}
  \end{align*}
Da $S_1 = S_2$, ist die Partitionierung nicht von $S$ nicht disjunkt.  Da
$(b_2, c_2) \in S$, aber $(b_2, c_2) \notin (S_1 \cup S_2)$, ist die
Partitionierung auch nicht vollständig.
\item
  \begin{align*}
    L \bowtie R &= \left(\bigcup_{i=1}^n R_n\right) \bowtie S =
    \bigcup_{i=1}^n\left(R_n \bowtie S\right) = \\
    &= \bigcup_{i=1}^n \left(R_n \bowtie \bigcup_{j=1} S_j \cup \left(S -
        \left(S \ltimes R\right)\right)\right) = \bigcup_{i=1}^n\bigcup_{j=1}^n
    \left(R_i \bowtie S_j\right)
  \end{align*}
  Da für $i \ne j$ gilt $R_i \bowtie S_j \subseteq R_i \bowtie S_i$, ergibt
  sich:
  \begin{align*}
    R \bowtie S &= \bigcup_{i=1}^n \left(R_i \bowtie S_i\right)
  \end{align*}
\item Die Formel gilt auch, wenn die Partitionierung über ein
  Nicht-Fremdschlüssel-Attribut durchgeführt wird.  Ein Fremdschlüssel zeichnet
  sich gegenüber einem normalen Attribut dadurch aus, dass er in der
  referenzierten Relation der Primärschlüssel ist und eindeutig und nicht null
  ist.  Null-Werte beeinflussen den Verbund allerdings nicht, und Duplikate
  bleiben bei der Partitionierung und beim Verbund erhalten.
\end{enumerate}

\end{document}
