\documentclass[11pt,a4paper]{scrartcl}
\usepackage[T1]{fontenc}
\usepackage[utf8]{inputenc}
\usepackage[ngerman]{babel}
\usepackage{microtype}
\usepackage{lmodern}
\usepackage{amsmath}
\usepackage{amsfonts}
\usepackage{amssymb}
\usepackage{enumerate}

\begin{document}

\author{Gruppe 14\\Max-Emanuel Hoffmann\\Ralf Vogler\\Sebastian Wiesner}
\title{Verteilte und Web-Informationssysteme}
\subtitle{Blatt 2}

\maketitle

\section*{Aufgabe 1}

\begin{enumerate}[a)]
\item Bind-Join:
\begin{align*}
t = (r + j)l + \frac{r*k_R + j*k_S}{b}
\end{align*}
Normaler Join:
\begin{align*}
t = 2l + \frac{s*k_S}{b}
\end{align*}
Es ist zu erkennen, dass die Latenzzeit beim \emph{Bind-Join} einen größeren
Einfluss hat, da sie proportional zur Größe der Relation $R$ in die Übertragung
eingeht, während sie beim normalen Join nur mit einem konstanten Faktor
einfließt.
\item Diese Situation kommt vor, wenn Daten aus Fremdsystemen gejoint werden
  müssen.  Eine Übertragung der Relation kommt beispielsweise nicht in Frage,
  wenn die andere Relation sehr groß ist, oder aus Gründen des Datenschutzes
  kein Zugriff auf alle Tupel gewährt werden darf.
\end{enumerate}

\section*{Aufgabe 2}

Eine Java-Implementierung dieser Aufgabe sowie zwei Dateien mit Test-Relationen
befinden sich im angehängten ZIP-Archiv.  Im Folgenden die Ausgaben der Klassen
\texttt{Test} und \texttt{Client}:

\begin{enumerate}
\item Dem Join-Operator ließe sich noch ein Parameter übergeben, welcher
  diejenige Spalte beschreibt, über die der Verbund berechnet werden soll.
\item Die angehängte Lösung verwendet zur Übertragung der Ergebnisse die
  Serialisierungsklassen der Java-API.  Damit lässt sich die Übertragung
  einfach implementieren, allerdings kann der Server nur schwer mit anderen
  Sprachen abgefragt werden.  Es wäre sinnvoller, die Ergebnisse über ein
  offenes Format wie XML oder JSON zu kommunizieren
\end{enumerate}

\subsection*{Ausgabe der Programme}

\minisec{Ausgabe von \texttt{Test}}

\begin{verbatim}
$ java -cp build/classes Test
--- Relation A ---
[id, name, zahl1, zahl2]
[1, Hans, 1.23, 3.14]
[2, Peter, 2.23, 4.14]
[3, Paul, 3.23, 5.14]

--- Relation B ---
[id, name, zahl3, zahl4]
[1, Hans, 1.23, 3.14]
[2, Peter, 2.23, 4.14]
[3, Johann, 3.23, 5.14]

--- A |><| B ---
[id, name, zahl1, zahl2, zahl3, zahl4]
[1, Hans, 1.23, 3.14, 1.23, 3.14]
[2, Peter, 2.23, 4.14, 2.23, 4.14]
\end{verbatim}

\minisec{Ausgabe von \texttt{Client} und \texttt{Server}}

\begin{verbatim}
$ java -cp build/classes Server &
[1] 23738
$ java -cp build/classes Client
--- A |><| B (remote) ---
[id, name, zahl1, zahl2, zahl3, zahl4]
[1, Hans, 1.23, 3.14, 1.23, 3.14]
[2, Peter, 2.23, 4.14, 2.23, 4.14]
$ kill 23738
\end{verbatim}

\end{document}
