\documentclass[11pt,a4paper]{scrartcl}
\usepackage[latin9]{inputenc}
\usepackage{ucs}
\usepackage{amsmath}
\usepackage{amsfonts}
\usepackage{amssymb}
\author{Max-Emanuel Hoffmann\\Ralf Vogler\\Sebastian Wiesner}
\title{Verteilte und Web-Informationssysteme WS11/12\\Blatt 2}

\begin{document}
\maketitle

\section{Aufgabe 1}
\subsection{a)}
Bind-Join:\\
\begin{align}
t = r*l + j*l + \frac{r*k_R + j*k_S}{b}
\end{align}
Normaler Join:\\
\begin{align}
t = s*l + \frac{s*k_S}{b}
\end{align}

\subsection{b)}
Wenn Daten von Fremdsystemen gejoint werden m�ssen. Eine �bertragung der gesamten Relation kommt z.B. nicht in Frage wenn die andere Relation sehr gro� ist oder aus Datenschutzgr�nden kein Zugriff auf alle Tupel gew�hrt werden darf.


\section{Aufgabe 2}
Java, siehe Anhang

\end{document}