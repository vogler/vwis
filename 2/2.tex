\documentclass[11pt,a4paper]{scrartcl}
\usepackage[T1]{fontenc}
\usepackage[utf8]{inputenc}
\usepackage[ngerman]{babel}
\usepackage{microtype}
\usepackage{lmodern}
\usepackage{amsmath}
\usepackage{amsfonts}
\usepackage{amssymb}
\usepackage{enumerate}

\begin{document}

\author{Gruppe 12\\Max-Emanuel Hoffmann\\Ralf Vogler\\Sebastian Wiesner}
\title{Verteilte und Web-Informationssysteme}
\subtitle{Blatt 2}

\maketitle

\section*{Aufgabe 1}

\begin{enumerate}[a)]
\item Bind-Join:
\begin{align*}
t = (r + j)l + \frac{r*k_R + j*k_S}{b}
\end{align*}
Normaler Join:
\begin{align*}
t = 2l + \frac{s*k_S}{b}
\end{align*}
Es ist zu erkennen, dass die Latenzzeit beim \emph{Bind-Join} einen größeren
Einfluss hat, da sie proportional zur Größe der Relation $R$ in die 
Übertragung eingeht, während sie beim normalen Join nur mit einem konstanten 
Faktor einfließt.
\item Diese Situation kommt vor, wenn Daten aus Fremdsystemen gejoint werden
müssen.  Eine Übertragung der Relation kommt beispielsweise nicht in Frage, 
wenn die andere Relation sehr groß ist, oder aus Gründen des Datenschutzes 
kein Zugriff auf alle Tupel gewährt werden darf.
\end{enumerate}

\section*{Aufgabe 2}
Java, siehe Anhang

\end{document}
